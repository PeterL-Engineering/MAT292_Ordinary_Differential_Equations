\documentclass{article} % For LaTex2e
\usepackage{iclr2022_conference,times}
% Optional math commands from https://github.com/goodfeli/dlbook_notation.
\input{math_commands.tex}

%######## MAT292: Uncomment your submission name
\newcommand{\matname}{ - Project Proposal}
%\newcommand{\matname}{Progress Report}
%\newcommand{\matname}{Final Report}

%######## MAT292: Put your Group Number here
%\newcommand{\gpnumber}{40}

\usepackage{hyperref}
\usepackage{xcolor}
\usepackage[normalem]{ulem}
\usepackage{url}
\usepackage{graphicx}
\usepackage{placeins}
\usepackage{float}
\usepackage{tikz}
\usepackage{multicol}

%######## MAT292: Put your project Title here
\title{Real-Time Neural Signal Filtering via \\
Hodgkin-Huxley Simulation Models}

%######## MAT292: Put your names, student IDs and Emails here
\author{\textbf{Peter Leong} \\
    Student\# 1010892955 \\
    peter.leong@mail.utoronto.ca
\And
    \textbf{Karys Littlejohns} \\
    Student\# XXXX \\
    karys.littlejohns@mail.utoronto.ca
\And
    \textbf{Katherine Shepherd} \\
    Student\# XXXX \\
    k.shepherd@mail.utoronto.ca
}



\iclrfinalcopy 
%######## MAT292: Document starts here
\begin{document}

\maketitle

% add vertical space between authors and abstract
\vspace{2ex}   % <-- adjust this value as needed


\begin{abstract}

%######## MAT292: Do not change the next line. This shows your Main body page count.
----Total Pages: \pageref{last_page}
\end{abstract}

\vspace{2ex}

\begin{multicols}{2}

\section{Introduction}

\section{Motivation \& Relevance}

In the field of biomedical engineering, extracting action potential timings from noisy extraceulluar recordings is essential in advancing brain-computer interfaces and neuroscience research.
While spike detection itself is a signal processing task, the underlying signal can be described and approximated by the Hodgkin-Huxley (H-H) model.
This set of non-linear ordinary differential equations models the ionic conductance changes that generate the action potential waveform.
Our challenge is to adapt the H-H waveforms to mimic the noise present in real-life data.

Beyond common EEG readings or microelectrode arrays, neural signal filtering and spike detection is relevant to other closed-loop systems such as epileptic seizure prediction \citep{addai-domfe2024epileptic} or adaptive deep brain stimulation for Parkinson's disease \citep{aljalal2022parkinson}.
Advancements in filtering are further motivated by the advent of high-density neural probes which generate large data streams requiring efficient, accurate processing solutions \citep{ye2024ultra}.
This project aims to develop a spike detection algorithm based on the H-H model to improve accuracy in low signal-to-noise ratio (SNR) environments.

\section{Scope \& Feasibility}

The scope of this project builds upon concepts from ESC103: Engineering Mathematics \& Computation and MAT292: Ordinary Differential Equations. 
The work is divided into three primary phases: (1) generating synthetic neural data by solving the Hodgkin-Huxley equations, (2) processing this data with a digital filter, and (3) developing a spike detection algorithm.

\subsection{Project Objectives}
The primary objectives of this project are:
\begin{enumerate}
    \item \textbf{Data Generation:} To implement numerical solvers for the Hodgkin-Huxley (H-H) model to generate realistic synthetic action potential data.
    \item \textbf{Signal Processing:} To design and apply a digital band-pass filter to isolate the spike waveform from the generated signal and added synthetic noise.
    \item \textbf{Spike Detection:} To develop an algorithm that detects action potentials using an adaptive threshold, calculated from the estimated noise floor of the processed signal.
    \item \textbf{Validation:} To qualitatively and quantitatively assess the performance of the detection algorithm on the noisy synthetic data.
\end{enumerate}

\subsection{Project Milestones \& Timeline}



\section{Technical Background}

\subsection{The Biological Basis: Ion Channels and Currents}
\subsubsection{The Cell Membrane and Resting Potential}
\subsubsection{Voltage-Gated Ion Channels}
\subsubsection{Sodium-Potassium Pump}

\subsection{Modeling Neurons as Electrical Circuits}
\subsubsection{The Parallel Conductance Model}
\subsubsection{Circuit Analogs: Capacitors, Resistors, and Batteries}

\subsection{Hodgkin-Huxley Equations}
\subsubsection{Model Components and Equations}
\subsubsection{Gating Variables and Activation/Inactivation}

\subsection{Numerical Methods}
\subsubsection{Euler's Methods}
\subsubsection{Runge-Kutta Methods}
\subsubsection{Comparison of Methods}


\section{Conclusion}

\label{last_page}

\newpage
\bibliographystyle{iclr2022_conference}
\bibliography{MAT292_Proposal_Ref}

\end{multicols}
\end{document}