\documentclass{article} % For LaTex2e
\usepackage{iclr2022_conference,times}
% Optional math commands from https://github.com/goodfeli/dlbook_notation.
\input{math_commands.tex}

%######## MAT292: Uncomment your submission name
\newcommand{\matname}{ - Project Proposal}
%\newcommand{\matname}{Progress Report}
%\newcommand{\matname}{Final Report}

%######## MAT292: Put your Group Number here
%\newcommand{\gpnumber}{40}

\usepackage{hyperref}
\usepackage{xcolor}
\usepackage[normalem]{ulem}
\usepackage{url}
\usepackage{graphicx}
\usepackage{placeins}
\usepackage{float}
\usepackage{tikz}
\usepackage{multicol}

%######## MAT292: Put your project Title here
\title{Real-Time Neural Signal Filtering via \\
Hodgkin-Huxley Simulation Models}

%######## MAT292: Put your names, student IDs and Emails here
\author{\textbf{Peter Leong} \\
    Student\# 1010892955 \\
    peter.leong@mail.utoronto.ca
\And
    \textbf{Karys Littlejohns} \\
    Student\# XXXX \\
    karys.littlejohns@mail.utoronto.ca
\And
    \textbf{Katherine Shepherd} \\
    Student\# XXXX \\
    k.shepherd@mail.utoronto.ca
}



\iclrfinalcopy 
%######## MAT292: Document starts here
\begin{document}

\maketitle

% add vertical space between authors and abstract
%\vspace{2ex}   % <-- adjust this value as needed


\begin{abstract}

%######## MAT292: Do not change the next line. This shows your Main body page count.
----Total Pages: \pageref{last_page}
\end{abstract}

\vspace{2ex}

\begin{multicols}{2}

\section{Introduction}
\label{sec: introduction}

\section{Motivation \& Relevance}
\label{sec: motivation_relevance}

In the field of biomedical engineering, extracting action potential timings from noisy extraceulluar recordings is essential in advancing brain-computer interfaces and neuroscience research.
While spike detection itself is a signal processing task, the underlying signal can be described and approximated by the Hodgkin-Huxley (H-H) model.
This set of non-linear ordinary differential equations models the ionic conductance changes that generate the action potential waveform.
Our challenge is to adapt the H-H waveforms to mimic the noise present in real-life data.

Beyond common EEG readings or microelectrode arrays, neural signal filtering and spike detection is relevant to other closed-loop systems such as epileptic seizure prediction \citep{addai-domfe2024epileptic} or adaptive deep brain stimulation for Parkinson's disease \citep{aljalal2022parkinson}.
Advancements in filtering are further motivated by the advent of high-density neural probes which generate large data streams requiring efficient, accurate processing solutions \citep{ye2024ultra}.
This project aims to develop a spike detection algorithm based on the H-H model to improve accuracy in low signal-to-noise ratio (SNR) environments.

\section{Scope \& Feasibility}
\label{sec: scope_feasibility}

The scope of this project builds upon concepts from ESC103: Engineering Mathematics \& Computation and MAT292: Ordinary Differential Equations. 
The work is divided into three primary phases: (1) generating synthetic neural data by solving the Hodgkin-Huxley equations, (2) processing this data with a digital filter, and (3) developing a spike detection algorithm.

\subsection{Project Objectives}
\label{subsec: project_objectives}

The primary objectives of this project are:
\begin{enumerate}
    \item \textbf{Data Generation:} To implement numerical solvers for the Hodgkin-Huxley (H-H) model to generate realistic synthetic action potential data.
    \item \textbf{Signal Processing:} To design and apply a digital band-pass filter to isolate the spike waveform from the generated signal and added synthetic noise.
    \item \textbf{Spike Detection:} To develop an algorithm that detects action potentials using an adaptive threshold, calculated from the estimated noise floor of the processed signal.
    \item \textbf{Validation:} To qualitatively and quantitatively assess the performance of the detection algorithm on the noisy synthetic data.
\end{enumerate}

\subsection{Project Milestones \& Timeline}
\label{subsec:milestones_timeline}

Although a more detailed outline of our project milestones and timeline can be found in \ref{app: appendix_a}, several high-level milestones are listed below.

\textbf{Week 4: } Finish implementation of Euler's and Improved Euler's method solvers. Also complete first iteration of noise generation algorithm.

\textbf{Week 6: } Finish implementation of Runge-Kutta method solver(s), and complete first iteration of band-pass filter.

\textbf{Week 8: } Refine band-pass filter, noise generation algorithm, and numerical methods.

\textbf{Week 10: } Discuss and evaluate results in the final report.

\textbf{Week 11/12: } Buffer time in case aforementioned tasks take longer than initially anticipated.


\section{Technical Background}
\label{sec:technical_background}

Understanding the bioligical model of neuron signals is imperative to acknowledging the abstractions and simplfications made to obtain the H-H equations.
In this section, we overview the basics behind action potential in the brain, and biochemical processes that occur to produce these phenomena.
Additionally, we outline the handful of key equations we intend to use from the H-H model, and discuss the theory behind the nuermical methods we have chosen.

\subsection{The Biological Basis: Ion Channels and Currents}
\label{subsec:biological_basis}
\subsubsection{The Cell Membrane and Resting Potential}
\label{subsubsec:resting_potential}
\subsubsection{Voltage-Gated Ion Channels}
\label{subsubsec:vgic}
\subsubsection{Sodium-Potassium Pump}
\label{subsubsec:na_k_pump}

\subsection{Modeling Neurons as Electrical Circuits}
\label{subsec:circuit_model}
\subsubsection{The Parallel Conductance Model}
\label{subsubsec:parallel_conductance}
\subsubsection{Circuit Analogs: Capacitors, Resistors, and Batteries}
\label{subsubsec:circuit_analogs}

\subsection{Hodgkin-Huxley Equations}
\label{subsec:hodgekin_huxley}
\subsubsection{Model Components and Equations}
\label{subsubsec:hh_equations}
\subsubsection{Gating Variables and Activation/Inactivation}
\label{subsubsec:gating_variables}

\subsection{Numerical Methods}
\label{subsec:numerical_methods}

Oftentimes in real-world modelling, differential equations do not have an algebraic solution.
This does not impede our ability to visualize or obtain results as we can instead turn to numerical methods to approximate the solution.
Within the field of numerical methods and computation, there are numerous methods to approximate a solution, however we will focus on three main methods: Euler's Method, Improved Euler's Method, and Runge-Kutta Methods.

\subsubsection{Euler's Method}
\label{subsubsec:euler_method}

Intuitively, Euler's Method solves the initial value problem by linking tangent lines for a finite number of time steps of size $\Delta t$.
Given a differential equation and initial condition:

\[
    \frac{dy}{dt} = f(t, y) \quad \text{where} \quad y(t_{0}) = y_{0}
\]

we can approximate the tangent line of the solution at the initial point $(t_{0}, y_{0})$ by:

\[
    y = y_{0} + f(t_{0}, y_{0})(t - t_{0})
\]

We approximate the solution at some $y_{i}$ by using the tangent line from the previous point (starting at $t_{0}$) to obtain more points.
As we increase the number of time steps and consequently decrease the size of $\Delta t$, we can improve our approximation of the solution curve.
Thus, we can generalize the approximation of the solution $\phi(t)$ with the following equation:

\[
    y_{n+1} = y_{n} + f(t_{n}, y_{n}) \cdot (t_{n+1} - t_{n})
\]

\subsubsection{Improved Euler's Method}
\label{subsubsec:improved_euler_method} % Removed space after colon

Though Euler's Method can give us a relatively concrete idea of the behaviour of the solution, it requires a large number of time steps to achieve accurate results.
Thus, many such methods aim to improve upon this base including Improved Euler's Method.
Intuitively, this method simply builds upon Euler's Method by considering the derivative at the subsequent point and computing the average.
We can define the general formula for the Improved Euler's Method with the following:
\[
    y_{n+1} = y_{n} + \frac{f_{n} + f(t_{n} + h, y_{n} + hf_{n})}{2}h
\]
such that 
\[
    h = t_{n + 1} - t_{n} \text{and} f_{n} = f(t_{n}, y_{n})
\]

\subsubsection{Runge-Kutta Methods}
\label{subsubsec:runge_kutta}

Euler's Method and Improved Euler's Method are technically considered Runge-Kutta class of numerical methods; however, in this section we will discuss the original methods developed by Runge and Kutta.
This method is classified as a fourth-order, four-stage Runge Kutta Method, but is commonly referred to as ``the'' Runge-Kutta Method.
Essentially, this method builds upon the two aforementioned methods by computing a weighted average of values at different points along the time step $\Delta t$.
The general form for the approximation of the solution is given by:
\[
    y_{n+1} = y_{n} + h \cdot \frac{k_{n1} + 2k_{n2} + 2k_{n3} + k_{n4}}{6}
\]
where we define each $k_{ni}$ as the following:
\[
    k_{n1} = f(t_{n}, y_{n})
\]
\[
    k_{n2} = f\left(t_{n} + \frac{1}{2}h, y_{n} + \frac{1}{2}hk_{n1}\right)
\]
\[
    k_{n3} = f\left(t_{n} + \frac{1}{2}h, y_{n} + \frac{1}{2}hk_{n2}\right)
\]
\[
    k_{n4} = f(t_{n} + h, y_{n} + hk_{n3})
\]

\section{Conclusion}

\label{last_page}

\newpage
\bibliographystyle{iclr2022_conference}
\bibliography{MAT292_Proposal_Ref}


\end{multicols}

\newpage
\appendix

\section{Gantt Chart}
\label{app: appendix_a}


\end{document}